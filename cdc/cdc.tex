\documentclass{article}

\usepackage[utf8]{inputenc}
\usepackage[T1]{fontenc}
\usepackage[frenchb]{babel}
\usepackage[top=3cm, bottom=3cm, left=3cm, right=3cm]{geometry}
\usepackage{hyperref}
\usepackage{graphicx}
\usepackage{fancyhdr}
\usepackage{fancybox}

\hypersetup{
	colorlinks=true
}

\title{Cahier des charges pour le jeu Cancer}
\author{Luc Chabassier <luc.linux@mailoo.org>}

\begin{document}
\maketitle

\tableofcontents

\section{Introduction}
Ce document doit servir de référence dans le développement du jeu Cancer. Les techniques de gameplay seront détaillées plus bas. Dans ce jeu, vous êtes une cellule cancéreuse qui veut rallier à sa cause un maximum d'autres cellules, avec pour objectif final la domination du corp. L'idée m'est venue en regardant ce \href{http://www.jeuxvideo.com/chroniques-video/00000345/3615-usul-jeux-inde-00000217.htm}{3615 Usul}, où l'idée est exprimée par le "développeur indépendant".

\section{L'histoire}
Vous êtes une gentille cellule qui fait son travail dans le corp, entourée de ses amies cellules. Mais un jour, un méchant agent mutagène viens vous transformer en une cellule cancéreuse. À cause de votre apparence désormais repoussante, vos amies cellules vous ignorent. Vous le vivez comme une trahison et décidez de toutes les faires muter à leur tour pour leur montrer qu'à l'intérieure vous êtes la même. Pour ce faire, tous les moyens sont bon, de la création d'une nouvelle amitié à la menace en passant par la force brutale. Pour finir le jeu, il faut conquérir l'intégralité du corp. Si vous y arrivez, l'humain dans lequel vous êtes se transforme en un monstre hideux.

\section{Le jeu}
C'est dans cette section que le gameplay sera décrit en détail.

\subsection{Le corp}\label{corp}
Le corp est décomposé en différents \hyperref[organe]{organes}. Au début, le jeu se déroule dans un seul \hyperref[organe]{organe}. Une fois que vous en avez pris le contrôle, vous avez accès aux \hyperref[organe]{organes} reliés à celui que vous contrôlez par des veines ou des artères. Et ainsi de suite : pour accéder à un \hyperref[organe]{organe}, vous devez contrôler un autre \hyperref[organe]{organe} relié par un vaisseau sanguin et envoyer des agents mutagènes par ce vaisseau sanguin. Envoyer un agent mutagène n'est pas gage de réussite : plus le vaisseau sanguin est long, plus ils ont de chance de se faire arrêter par les anticorps. De plus, certains \hyperref[organe]{organes} sont mieux défendus que d'autres. Enfin, le sens du vaisseau sanguin est important : il doit aller de votre \hyperref[organe]{organe} à celui que vous ciblez pour être empruntable. Le contrôle de tous les \hyperref[organe]{organes} signifie la fin du jeu.

\subsubsection{Liste d'\hyperref[organe]{organes}}
Les \hyperref[organe]{organes} utilisés seront :
\begin{itemize}
	\item Cerveau (\emph{brain})
	\item Cœur (\emph{heart})
	\item Poumons (\emph{lung})
	\item Foie (\emph{liver})
	\item Rein (\emph{kidney})
	\item Système Digestif (\emph{digestive})
\end{itemize}

\subsubsection{Graphiques}
Le corp est un schéma dans lequel apparaissent les organes et les liens les unissant (les veines et artères).

\subsubsection{Sélection}
Un organe sélectionné est entouré de couleur. Les collisions se font au pixel près en se basant sur le masque alpha des images.

\subsubsection{Chargement}
Le corp est une image du nom \emph{corpse.png}. Chaque organe est constitué de trois images : une quand il est contrôlé, une quand il est contaminé et une quand il est sain. Chaque image doit faire la même taille que celle du corp, mais son fond doit être transparent. Les nom des images doivent être \emph{organe/état.png}, avec état qui peut valoir \emph{control}, \emph{sick} et \emph{healthy}. Il y a donc un dossier par organe.
Par exemple, pour le \emph{cerveau}, on aura :
\begin{verbatim}
brain/control.png
brain/sick.png
brain/healthy.png
\end{verbatim}
Tout ceci est placé dans un dossier (\emph{organs/} par défaut).

\subsection{Un organe}\label{organe}
Chaque organe est composé de \hyperref[cellule]{cellules}. Au départ, vous n'en possédez qu'une seule. Vous devez faire muter les autres \hyperref[cellule]{cellules}. Vous faire muter une \hyperref[cellule]{cellule}, vous devez prendre contact avec elle. Vous avez plusieurs possibilités. D'abord, si vous êtes en contact avec elle, vous pouvez tenter le \hyperref[dialogue]{dialogue}, la \hyperref[force]{force} ou encore \hyperref[intimi]{l'intimidation}. Le principe consiste à chaque fois à faire avaler des agents mutagènes à la \hyperref[cellule]{cellule}. Ensuite, si vous êtes reliés à elle par un vaisseau sanguin qui va de vous vers elle, vous pouvez tenter l'envoie de médiateur et d'agent mutagènes. Les médiateurs envoyés vont se fixer sur une \hyperref[cellule]{cellule} au hasard qui touche le vaisseau sanguin. Une fois qu'il a rejoint la \hyperref[cellule]{cellule}, vous pouvez ou tenter le \hyperref[dialogue]{dialogue} ou \hyperref[intimi]{l'intimidation}. Une fois la \hyperref[cellule]{cellule} convaincue, le médiateur va capter les agents mutagènes passant par le vaisseau sanguin jusqu'à mutation de la \hyperref[cellule]{cellule}.

Pour chaque organe, vous possédez un capital en nutriments et en oxygène. Vous en gagnez en le captant aux \hyperref[cellule]{cellules} voisines et aux vaisseaux sanguins. Ce capital sert à nourrir vos \hyperref[cellule]{cellules}, à créer des agents mutagènes et à envoyer des médiateurs. Pour prendre le contrôle d'un organe, il faut contrôle plus de 90\% des \hyperref[cellule]{cellules}.

\subsubsection{Graphiques}
Chaque \hyperref[cellule]{cellule} est dessinée sous la forme d'un hexagone, et les vaisseaux sanguins sont des ensembles joints d'hexagones.

\subsubsection{Interface}
La tilemap remplit tout l'écran. Il y a un container dans un coin qui affiche l'oxygène et les nutriments restants. Lorsque qu'une cellule est sélectionnée, un autre container est affiché, qui contient plusieurs informations sur la cellule :
\begin{itemize}
	\item Si elle n'est pas contaminée, on affiche son taux d'acceptation, sa défense restante, si on est en contact avec elle, combien de médiateurs sont sur elle et combien d'unités d'attaque sont en train de l'attaquer.
	\item Si elle est contaminée, on affiche sa consommation de nutriment (positif = elle en crée, négatif = elle en utilise) et sa consommation en oxygène (idem). On affiche aussi le nombre de médiateurs, d'unités d'attaque qu'elle doit produire et un bouton pour l'hiberner. Si la cellule est en hibernation, on n'affiche qu'un seul bouton, pour la réveiller.
\end{itemize}

\subsubsection{Contrôle}
On clique sur un cellule pour la sélectionner. Si on fait un clic droit sur une cellule non contaminée alors qu'une cellule contaminé est sélectionnée, on propose d'envoyer des unités d'attaque si les deux cellules sont adjacentes et on propose d'envoyer des médiateurs si les deux cellules ont un lien (contact ou vaisseau sanguin).

\subsubsection{Stockage}
Chaque case de l'hexamap est stockée sous la forme : \emph{type:autre} où type peut être :
\begin{description}
	\item[empty] Correspond à un case vide. \emph{autre} ne contiendra rien.
	\item[vessel] Correspond à un vaisseau sanguin. Voir la description des vessels pour le contenu de \emph{autre}.
	\item[cell] Correspond à une cellule. Voir la description des cellules pour le contenu de \emph{autre}
\end{description}

\subsection{Une cellule}\label{cellule}
Les cellules sont les unités de bases du jeu. Les cellules mutées vous appartiennent et vous devez faire muter les autres. Chaque cellule mutée consomme régulièrement un peu de nutriments (20 nutriments par minute), mais peu aussi en faire gagner lorsqu'elle en prend aux veines ou aux cellules non mutées. Une cellule mutée complètement entourée par des cellules mutées et qui ne touche pas de vaisseau sanguin vera sa consommation en nutriments divisée par deux. Si vous n'avez pas assez de nutriments pour toutes vos cellules, vous devez en mettre certaines en hibernation : elles ne consommeront plus rien mais ne pourront plus rien faire. Pour les faire sortir d'hibernation, vous devrez consommer 20 oxygènes. Si toutes vos cellules sont en hibernation, vous avez perdu. Chaque action (créer,dialoguer...) de vos cellules consomme de l'oxygène.

\subsubsection{Méthodes de mutation}\label{muta}
Pour faire muter une cellule non mutée, vous devez lui faire accepter la mutation, puis lui envoyer autant d'agents mutagène que nécessaire. Pour la faire passer en état d'acceptation, vous avez trois possibilités. Pour chacune d'entre elle, le calcul sera donné, en utilisant les variables $t$, le taux d'acceptation de 0 à 100 de la cellule, et $d$, la défense restante de 1 à 100.
\begin{description}
	\item[Dialogue]\label{dialogue}
		Le dialogue consiste en une suite de mini-jeux qui vous accordent des points : plus vous avez de points, plus vous convainquez la cellule cible. C'est la méthode la moins coûteuse : elle consomme juste un peu d'oxygène. Plus le taux d'acceptation est élevé au début du dialogue, plus le dialogue le fera augmenter. Par contre, le dialogue sera moins efficace sur une cellule dont la défense est entamée.\\
		\begin{description}
			\item[Coût] Coûte 20 oxygènes par mini jeu fait.
			\item[Calcul ($n$ est le nombre de points obtenus)] $v_{acceptation} = (n*t) - (100-d)$ -> zéro minimum.
		\end{description}

	\item[Force]\label{force}
		La force est une méthode très coûteuse. Elle nécessite la création d'unités d'attaque, coûteuses en nutriments et en oxygène. De plus, elle ne fonctionne qu'avec les cellules avec lesquelles vous êtes directement en contact. Enfin, vous n'avez aucune idée de la défense de la cellule cible avant d'avoir commencé l'attaque. Tant que durera l'attaque avant les 50\% de la défense, le taux d'acceptation baissera. Au contraire, si vous dépassez ce niveau, vous ferez augmenter ce taux. Si vous brisez complètement les défenses, la cellule cible passe en d'état d'acceptation immédiatement. Attaquer une même cellule avec plusieurs à la fois permet de gagner du temps, mais c'est très coûteux.
		\begin{description}
			\item[Coût] Chaque unité d'attaque coûte 30 oxygènes et 40 nutriments.
			\item[Temps de création] Une unité d'attaque met 30 secondes à se créer.
			\item[Durée de vie] Une unité d'attaque vit 45 secondes.
			\item[Calcul ($n$ est le nombre d'unités d'attaque)] Variation acceptation et défense :\\
				\begin{itemize}
					\item $v_{defense} = int(n/2) * -1$ par seconde.
					\item $v_{acceptation} = if(d>50) then\{ [(d-50) / 25] * -1 \} else\{ (50-d) / 25\}$ par seconde.
				\end{itemize}
		\end{description}

	\item[Intimidation]\label{intimi}
		L'intimidation est une méthode intermédiaire. Elle consiste en l'envoi de médiateurs. Il faut en envoyer plus d'un (un seul servant à tenter le dialogue avec une cellule distante). L'intimidation ne peux pas faire baisser le taux d'acceptation. Par contre, elle est plus efficace si ce taux est faible. Plus vous envoyez de médiateurs, plus l'intimidation est efficace. Elle est aussi plus efficace sur une cellule dont la défense est entamée.
		\begin{description}
			\item[Coût] Chaque médiateur coûte 20 oxygènes et 5 nutriments.
			\item[Temps de création] Un médiateur met 10 secondes à se créer.
			\item[Durée de vie] Un médiateur vit 2 minutes.
			\item[Calcul ($n$ est le nombre de médiateur)] $v_{acceptation} = [int(n/4) * ((100-d) / 100)] - (t/4)$ -> zéro minimum par seconde.
		\end{description}
\end{description}

\subsubsection{Graphiques}
Chaque cellule est un hexagone, de couleur différente en fonction de si elle est saine, mutée ou en hibernation. Les unités d'attaques sont de petits crabes. Elles ne sont pas forcément tous affichés (ça ne sert à rien d'en afficher vingt sur 10 pixels). Le médiateurs sont de petites boules tentaculaires. Comme pour les unités d'attaque, les médiateurs ne sont pas forcément tous affichés.

\subsubsection{Stockage}
Chaque cellule est stockée dans une chaine de caractères de la forme \emph{[muted|sane]:autre}. Le début indique si la cellule est mutée et le \emph{autre} en dépend. % TODO autre en fonction de l'état de la cellule.

\subsection{Sons}
Le jeu possède deux musiques : une pour le menu et une pour le jeu lui-même. Il possède aussi un certain nombre de sons que voici (avec leurs noms) :
\begin{description}
	\item[ok] Son est joué lors des validations.
	\item[cancel] Son joué lors des annulations.
	\item[select] Son joué lors d'une sélection.
	\item[wincel] Son joué lors de la mutation d'une cellule.
	\item[winorg] Son joué lors de l'obtention du contrôle d'un organe.
	\item[alert] Son joué lors d'un problème.
	\item[hibern] Son joué lors de la mise en hibernation d'une cellule.
\end{description}
Les sons et musiques sont contenus dans un dossier qui contient deux dossiers : un 'music' et un 'sounds'. Le 'music' contient les musiques 'game.ogg' et 'menu.ogg'. Le dossier 'sounds' contient les sons énumérés ci-dessus au format wav, ainsi qu'un fichier texte nommé 'list' qui sur chaque ligne contient une des catégories citées ci-dessus, le nom d'un fichier son, et le canal sur lequel le lancer (0 étant le premier). Le tout séparer par des espaces. Exemple :
\begin{verbatim}
ok ok.wav 0
cancel cancel.wav 0
select nom_idiot.wav 0
wincel tadada.hop 1
winorg autrechose.wav 1
hibern hibern.wav 1
alert beep.wav 2
\end{verbatim}
Seuls les .wav sont supportés.

\subsection{Configuration}
Vous pouvez changer la résolution, l'état plein écran et le thème sonore.

\section{Programmation}
Ce jeux sera programmé en c++. Les bibliothèques utilisées seront la \href{http://www.libsdl.org/}{SDL}, \href{https://github.com/lucas8/SDLP\_position}{SDLP\_position}, \href{https://github.com/lucas8/SDLP\_tools}{SDLP\_tools} pour les graphismes, 
\href{https://github.com/lucas8/SDLP\_event}{SDLP\_event} pour les évènements, 
\href{http://www.gnu.org/software/gettext/}{GNU gettext} pour la traduction, 
\href{http://www.libsdl.org/projects/SDL\_mixer/}{SDL\_mixer} pour la musique et les sons, 
\href{http://www.ferzkopp.net/joomla/content/view/19/14/}{SDL\_gfx} pour les zooms et rotations de surface,
\href{http://www.boost.org/}{boost} pour les chemins de fichiers, les pointeurs de fonction et les expressions régulières, 
\href{http://www.grinninglizard.com/tinyxml/index.html}{tinyXml} pour les fichiers xml, 
et \href{http://guichan.sourceforge.net/wiki/index.php/Main\_Page}{guichan} pour la GUI. Le système d'animation et d'abstraction des SDL\_Surface sera codé directement dans le jeu. Le versioning sera réalisé avec \href{http://git-scm.com/}{git} et le jeu sera hébergé sur \href{https://github.com/}{github}. Étant moi-même sous \href{https://fr.wikipedia.org/wiki/Linux}{GNU/Linux} (\href{http://www.archlinux.org/}{ArchLinux} plus précisément), le jeu ne sera au départ disponible que pour cet OS. Le portage vers Windows et Mac se fera sans doute ultérieurement.

\section{Conclusion}
Ce document n'est pas encore complet. De plus, on peut imaginer de nombreuses évolutions au jeu, mais coder tout ceci représente déjà un sacré travail.

\end{document}


